% LAST UPDATE: 13.12.2012 % 


\subsection{Timed Input/Output Automata}
\label{background-tioa}

The "Timed Input/Output Automata" is a basic, mathematical specification
framework for description and analysis of real time systems.  In this framework,
system is represented by non-deterministic, possibly infinite-state, state
machine referred as “timed I/O automaton” (TIOA)
\cite{Kaynar:2006:TTI:1203437}. TIOA has been implemented as the modeling
language in ECDAR \cite{conf/atva/DavidLLNW10}.

\begin{figure}[t]
\label{simple-model}
\begin{centering}
\includegraphics[scale=0.6]{images/simplefied_uni}
\par\end{centering}
\caption{Model of beverage-serving machine and researcher.}
\label{bev-machine}
\end{figure}

The preceding figure (see fig. \ref{simple-model}) illustrates the system
consisting of two automatons: \emph{Coffee Machine} and \emph{Researcher}.  The
\emph{Coffee Machine}, given a coin (\emph{coin}), it serves either coffee
(\emph{cof}) or tea (\emph{tea}) to the \emph{Researcher} within a given time
interval. Moreover, free tea is served once in a while. The \emph{Researcher} is
producing publications (\emph{pub}), once provided a timely stimuli in form of
preferred beverage (\emph{cof}).  In the example, the \emph{Coffee Machine} -
TIOA consists of and two locations represented by circles: \emph{Idle} and
\emph{Serving}. \emph{Idle} represents the starting location, and the state of
machine waiting for coin input (\emph{coin?}). Analogously, the
\emph{Researcher} is in \emph{Idle} state expecting either coffee (\emph{cof})
or tea (\emph{tea}) provided by \emph{Coffee Machine}.  The flow of each TIOA is
controlled by three types of labels: \emph{invariants}, \emph{guards} and
\emph{clock-reset operations}.  Invariants are defined on locations ($y\leq 6$
and $x\textgreater 4$) and represent constraints for the clocks in order for the
control to remain in particular location until time requirement is fulfilled.
Guards are located on the edges ($y\geq 2$ and $y\geq 4$) and express conditions
on the values of clock that must be satisfied in order for the edge to be
taken. When the condition is satisfied, the transition occurs and action
(\emph{cof!}, \emph{tea!} or \emph{pub!}) is triggered. Clock-reset operations
($y=0$) are simple clock value manipulations in form of assignment that enforce
progress in the system.

In ECDAR, the specification interface is leveraging the UPPAAL TIGA language
\cite{behrmann_uppaal-tiga:_2006} to describe TIOA. However, the following
constraints are retained\footnote{See
  \url{http://people.cs.aau.dk/adavid/ecdar/examples.html#lang}}:


\begin{itemize}
\item Invariants may not be strict.
\item Inputs must use controllable edges.
\item Outputs must use uncontrollable edges.
\item All channels must be declared broadcast.
\item The system is implicitly input enabled due to broadcast communication but
  for refinement checking purposes the relevant inputs must be explicit in the
  model.
\item In the case of parallel composition of several components, a given output
  must be exclusive to one component.
\item For implementations, outputs must be urgent.
\item For implementations, every state must have independent time progress,
  i.e., progress must be ensured by either an output or infinite delay.
\item $\tau$-transitions (no output or input) are forbidden.
\item Global variables are forbidden.
\end{itemize}

\subsection{Code Generation}
\label{background-codegeneration}

In order to clarify what code generation is one need to understand what a model
transformation is, as this is a fundamental part of code generation. In short
one could say that the model transformation is a way to ensure that the final
code is consistent and with a reduced number of errors. The generation is an
automated way to produce code from models. The actual generation is defined by
the software developer, thus it is defined what the output should be, but the
input and the data is not.

There is generally two ways to do model transformations, that is model to model
and model to text, the former known as M2M and the latter M2T. There are also a
lot of other tools and techniques for transformation, which should not be
confused with model transformations. One could mention an XSLT-transformation as
an example, where the base input is an XML-document and the final output is
another XML-document, often XHTML, with a predefined XML-Schema.

Model to model is a transformation of a number of models to a given number of
new models - from X number of models to Z number of new models. Model to text is
the transformation of a number of models to text, the text could for instance be
code - which is why the process sometimes is known as model to code.
