This paper present a framework for code generation, based on timed automatas
modelled in ECDAR (Environment for Compositional Design and Analysis of Real
Time Systems). A graphical tool based on UPPAAL that allows to visually create
models of real-time systems.

A key feature missing in ECDAR is the possibility to utilize models for code
generation, as way to develop software solutions based on visually represented
models.

This project introduces how java code is generated from an ECDAR model. It also
introduce the notation of tasks as an extension to the ECDAR model within the
framework.

The functionality of the framework is evaluated through a series of test. All of
which is based on beverage-serving machine model (See figure \ref{bev-machine}
on page \pageref{bev-machine}).


\todo{Introduce tasks:
At man kan placere en opgave paa forskellige lokationer.
}
\todo{Define problemstatement:
A kunne udnytte modeller til at genere code til en 
}
\todo{Tell about our solution:
Vores framework der generere java code ud fra automatas i ecdar filen
}
